% ----------------------------------------------- %
% --- File: IEEE_paper.tex  --------------------- %  
% --- Authors: Nick, Jed, Zach, K-Dink   -------- %
% --- Due Date: 11:59PM, Monday, October 29th --- %
% ----------------------------------------------- %

% IEEE Special Document Class
\documentclass[twocolumn,letterpaper]{IEEEAerospace2012}

% Other packages
\usepackage[pdftex,final]{graphicx}
\usepackage{amsmath,amssymb}

% New commands
\newcommand{\ignore}[1]{}  % {} empty inside = %% comment

% Scientific notation:
\providecommand{\e}[1]{\ensuremath{\times 10^{#1}}}

% General
\newcommand{\parens} [1] {\left(  #1  \right)}
\newcommand{\brackets} [1] {\left[ #1 \right]}
\newcommand{\rootdir}{./Figures/}

% Array
\newcommand{\arrayp}[2]{\parens{ \begin{array}{#1}  #2 \end{array} } }
\newcommand{\arrayb}[2]{\brackets{ \begin{array}{#1}  #2 \end{array} } }

% Multiletter Variables
\newcommand{\RN}{\mathit{RN}}
\newcommand{\DC}{\mathit{DC}}
\newcommand{\SNR}{\mathit{SNR}}

% -------------- $
% -- DOCUMENT -- $
% -------------- $
\begin{document}

\title{DayStar: Modeling and Test Results of a Balloon-Borne Daytime Star Tracker}

\author{%
    N. Truesdale, K. Dinkel, Z. Dischner, J. Diller \\
    Aerospace Engineering Sciences\\
    University of Colorado at Boulder\\
    Boulder, CO 80309\\
    \and 
    Dr. Eliot Young\\
    Space Studies Department\\
    Southwest Research Institute\\
    Boulder, CO 80302
    %
    \thanks{\footnotesize 978-1-4577-0557-1/12/$\$26.00$ \copyright2012 IEEE.}              % This creates the copyright info that is the correct 2012 data
    %\thanks{{978-1-4577-0557-1/12/$\$26.00$ \copyright2012 Crown.}}          % Use this copyright notice only if you are employed by a crown government (e.g., Canada, UK, Australia)
    %\thanks{{U.S. Government work not protected by U.S. copyright.}}         % Use this copyright notice only if you are employed by the U.S. Government.
    %%%%% Use this footnote to keep track of your paper number/versioning.
    \thanks{$^1$ IEEEAC Paper \#xxxx, Version x, Updated dd/mm/yyyy.} % NOTE: Modify this line to reflect your paper number, version, and date of version for tracking purposes.
}


\maketitle

\begin{abstract}
    High altitude balloons are capable of supporting astronomical observations with virtually no image degradation due to atmospheric turbulence. To take advantage of this space-like seeing, a telescope must be pointed and stabilized with sub-arcsecond precision. This problem consists of two parts: providing an error signal, and using it to correct the pointing. This paper addresses error signal acquisition, specifically focusing on modeling and flight testing of the DayStar star tracker.

    DayStar is a star tracker designed under the University of Colorado Aerospace Capstone Program with support from Southwest Research Institute. It is intended to improve upon the pointing accuracy and daytime performance of the ST5000, a star tracker commonly used in NASA’s sounding rocket program. The ST5000 was shown to work on a balloon at night, but failed to acquire stars during daytime. DayStar remedies this issue by filtering light below 620 nm and by using a CMOS sensor with high red-performance and resolution. This attenuates most of the sky background, which, combined with custom star identification algorithms, allows stars be seen during the day. 

    To validate modeling and demonstrate daytime star acquisition, a DayStar prototype flew on a high altitude balloon in September, 2012. The filtered camera typically saw four or more stars during daytime, proving the ability to operate diurnally. This paper will further discuss DayStar’s ability to obtain a Lost-in-Space solution during daytime as a function of sky background and galactic latitude of the field of view. It will also focus on the precision of star centroiding algorithms and the pointing acuity for both day and night conditions. These findings will be used to validate the performance model and examine DayStar as a potential star tracker for high altitude balloon observatories. 
\end{abstract}

\tableofcontents

%-------------------------------------------------------------------------------------------------------
\section{Introduction}
This section will introduce the daytime startracker problem, and give the background and motivation necessary to setup our prototype and the importance of its performance.

%-------------------------------------------------------------------------------------------------------
\section{Modeling}
The modeling for DayStar is split into two parts: image quality and algorithm performance. The two are functionally independent, and must be linked by one or more parameters. The primary link is signal-to-noise ratio ($\SNR$), though the shape of stars in the image is also important. The following sections describe how a threshold SNR and star shape are set by the algorithms analysis, and how image quality analysis is then used to support DayStar's design parameters and predict daytime performance.

\subsection{Algorithms Analysis}
KEVIN, you write here.

\subsection{Image Quality Analysis}
Given a minimum SNR and a star blur of 4 to 6 pixels square, the goal of the image modeling is to determine the number of stars visible during night and day. This requires identification of signal and noise sources; we assume that noise is generated by the sky background and by stars themselves, as well as by dark current ($\DC$) and read noise ($\RN$) in the image sensor. SNR is given by Equation \ref{eq:snr}, in which $t$ is exposure time and $n$ is the number of pixels in a frame encompassing a star.  
\begin{equation} 
    \label{eq:snr}
    \SNR = \frac{F_{star}t}{\sqrt{F_{star} t + F_{sky} n t + \DC n t + \RN^2 n }}
\end{equation}


%-------------------------------------------------------------------------------------------------------
\section{Test Flight Results}
This is the money section. We will summarize our test flight (very briefly) and then launch into our results. Again, the focus will be on daytime, and we will hammer home how well we see stars and track during the day.

\subsection{Flight and Concept of Operations}
On September 22nd, 2012 the DayStar system flew aboard the Wallops Arc Second Pointer (WASP) platform as a secondary payload. The flight was launched out of Fort Sumner, NM as part of the CSBF 2012 campaign. The entire flight lasted about 15 hours starting at 8:30 am. During this flight DayStar actively operated for 5.5 hours starting operation 2.5 hours before a modelled astronomical sunset at an altitude of 35 km and ending around 17:30 mountain standard time. Operation was split into three separate modes: daytime, twilight, and nighttime. During daytime and nighttime operation, bursts of images taken at 10Hz for durations of 20 and 50 seconds respectively were captured. These bursts were taken with varying camera settings (gain and exposure time). During twilight operation 5 seconds bursts were taken and the camera settings were changed more rapidly to capture the effect of the varying background brightness. In total some 460 GB of images and system health and status data was collected. DayStar was mounted on the aft of the WASP platform such that when the WASP payload was under pointing control, DayStar was pointed near perpendicular to the sun and its field of view was centered 45 degrees above horizontal.

\subsection{Analysis}
Cool graphs:
1. Star path comparison of frame 0 and 100 frames later
2. Show different centroiding locations on same image
3. Number of stars found, matched for daytime and nighttime  
4. tracking performance plots: yaw, pitch, roll, for day and night

\subsection{Tracking Performance}
Without a true attitude solution, it is still possible to obtain a performance metric for how well the DayStar tracks stars. This is done with a purely numerical analysis.

Given a set of images, the first task is to obtain a list of star centroid locations. By comparing the stars in each image to a reference (epoch) set of star vectors, in this case, just the first image in a set, the motion in said star locations can be described as a set of rotations. The "q-method" is a widely used method to determine rotations based on minimizing the rms of the residuals between coordinate frames. It was employed here to find the quaternion rotation between the reference frame and each subsequent.

Quaternion rotations were then converted into the Euler angles: roll,pitch, and yaw. Each set of angles contain information about the change in star locations in time. The predominant modes of change are low-frequency rotations, and high frequency rotations. The low frequency observations are due to the motion of the gondola, while high frequency observations are due to the inaccuracies of the DayStar system. 

A measure of the system's precision is desired, so the low-frequency rotations need to be removed. This was done by examining the rotations in the frequency domain, using fast fourrier transforms, and applying a high-pass filter to the transformed rotations. Transforming back into the time domain, the rms of each the resulting rotation sets describes DayStar's ability to identify and track stars over an image set. 

  

%-------------------------------------------------------------------------------------------------------
\section{Conclusions}
Here we will inspiringly conclude by tying our results and modeling together, and saying why this is so important to the future of balloon science.

%-------------------------------------------------------------------------------------------------------
\section{References}


%-------------------------------------------------------------------------------------------------------
\section{Biography}

\appendices

\acknowledgments
The authors thank the Office of Naval Research for funding this project.


\thebiography
\begin{biographywithjpg} 
{Nick Truesdale}{\rootdir Nick.jpg}
is a graduate student in Aerospace Engineering at the University of Colorado.
He has been the Electrical Systems Lead for four balloon payloads, and has a wealth of experience designing power sytems and embedded electronics. His other projects have included CubeSats with the University of Colorado and radar scattering simulation with MIT Lincoln Laboratory. Nick enjoys playing guitar and marimba as well as downhill skiing.   
\end{biographywithjpg}

\begin{biographywithjpg} {Kevin Dinkel}{\rootdir Kevin.jpg}
is 
\end{biographywithjpg}

\begin{biographywithjpg} {Zach Dischner}{\rootdir Zach.jpg}
is
\end{biographywithjpg}

\begin{biographywithjpg} {Jed Diller}{\rootdir Jed.jpg}
is
\end{biographywithjpg}

\begin{biographywithjpg} {Eliot F. Young}{\rootdir Eliot.jpg}
Eliot F. Young received an A.B .in
Physics from Amherst College in 1984,
an S.M. in Aeronautical Engineering
from M.I.T. in 1987, an S.M. in Earth,
Atmospheric and Planetary Science
(EAPS) from M.I.T. in 1990, and an
Sc.D. from M.I.T. (EAPS) in 1992. He is
currently a Principal Scientist at
Southwest Research Institute in Boulder,
CO, in the Department of Space Studies. His current areas
of study include the surfaces and atmospheres of Pluto, Tri-
ton, Eris and other large Trans-Neptunian Objects, as well
as the distributions of aerosols and trace gases in Titan's
atmosphere and the wind fields on Venus. He has led ob-
serving campaigns on four continents.
\end{biographywithjpg}

\end{document}